\documentclass[
	%draft,
	submission,
	%compressed,
	final,
	%
	%technote,
	%internal,
	%submitted,
	%inpress,
	%reprint,
	%
	%titlepage,
	notitlepage,
	%anonymous,
	narroweqnarray,
	inline,
	twoside,
  %invited,
	]{ieee}

\newcommand{\latexiie}{\LaTeX2{\Large$_\varepsilon$}}

%\usepackage{ieeetsp}	% if you want the "trans. sig. pro." style
%\usepackage{ieeetc}	% if you want the "trans. comp." style
%\usepackage{ieeeimtc}	% if you want the IMTC conference style
\usepackage{graphicx}
\usepackage{cite}
\usepackage{url}
\usepackage{amsmath}
% Use the `endfloat' package to move figures and tables to the end
% of the paper. Useful for `submission' mode.
%\usepackage {endfloat}

% Use the `times' package to use Helvetica and Times-Roman fonts
% instead of the standard Computer Modern fonts. Useful for the 
% IEEE Computer Society transactions.
%\usepackage{times}
% (Note: If you have the commercial package `mathtime,' (from 
% y&y (http://www.yandy.com), it is much better, but the `times' 
% package works too). So, if you have it...
%\usepackage {mathtime}

% for any plug-in code... insert it here. For example, the CDC style...
%\usepackage{ieeecdc}

\begin{document}

%----------------------------------------------------------------------
% Title Information, Abstract and Keywords
%----------------------------------------------------------------------
\title{Face Obscuration using Big Data Processing Techniques}

% format author this way for journal articles.
% MAKE SURE THERE ARE NO SPACES BEFORE A \member OR \authorinfo
% COMMAND (this also means `don't break the line before these
% commands).
\author[EILAR AND VENKATESH]{Cody W. Eilar
       \authorinfo{C.\,W.\,Eilar is with the Department of Computer
       Engineering, University of New Mexico, Albuquerque, NM 87109.
       Phone: $+$1\,505\,514-5961, e-mail: ceilar@unm.edu}%
       \and{} Venkatesh Jatla 
       \authorinfo{V.\, Jatla is with the Department of Computer
       Engineering, University of New Mexico, Albuquerque, NM 87109.
       Phone: $+$1\,505\,908-7361, e-mail: venkatesh369@unm.edu}
}

\journal{CS 567, Big Data}
\titletext{ \today \ Dr. Trilce Estrada}
%\ieeecopyright{0018--9456/97\$10.00 \copyright\ 1997 IEEE}
%\lognumber{xxxxxxx}
%\pubitemident{S 0018--9456(97)09426--6}
%\loginfo{Manuscript received September 27, 1997.}
\firstpage{1}

%\confplacedate{Ottawa, Canada, May 19--21, 1997}
\maketitle               

%----------------------------------------------------------------------
% ABSTRACT
%----------------------------------------------------------------------
\begin{abstract} 
  With big data processing techniques and tools becoming ubiquitous, it
  is now more feasible than ever to process images and videos over 
  distributed computing environments using tools such as Hadoop \cite{hadoop}, Spark \cite{spark}, Storm \cite{storm} and others. 

  In this paper, we investigate 
  using these "big-data" techniques to perform face obscuration on very large
  video and image datasets collected by the Advancing Out of School Learning in 
  Mathematics and Engineering (AOLME) program so that these datasets can be redistributed for further
  investigation by researchers who do not have special permissions to view 
  them. By removing the faces in these videos, researchers who are focused
  on gestures and actions can work with data without having to be 
  part of any institutional review board (IRB) process.  
\end{abstract}


%----------------------------------------------------------------------
% SECTION Background and Theory
%----------------------------------------------------------------------
\section{Background}
\PARstart The advancing out of school learning in mathematics and engineering (AOLME) is program designed to help under 
represented middle school students to become interested in science, technology, engineering and mathematics (STEM). The goal of
the program is to not only provide these students with access to high quality programs, but to also help educators analyze
what learning methods work well in the classroom \cite{aolme_paper}. Using video cameras in the classroom, AOLME attempts to capture 
thousands of hours of student to student and student to facilitator interactions. The goal is then to analyze these videos
to further aid educators understand what is working in the classroom and what is not, i.e. when students are learning or when they are not. 
The problem with this currently is that
all the videos must be annotated by hand which is very time consuming and tedious. To aid educators with their research, 
it would be extremely useful to have automated methods for annotating videos using machine learning and 
video processing techniques. Because AOLME has a vast array of videos, it seems that "big data" techniques are a
good fit for solving the problem of autonomous video annotation.

The list of potential features that can be extracted from this video data is vast, we
therefore need to narrow the scope of features for
which we can track to a limited subset. Due to restrictions of the IRB, 
it was natural to select features which did not involve the students faces
such as when a student is typing or writing. Since these are the desirable 
features, we no longer needed to have the faces of the students in the videos
and in order to collaborate with other researchers that do not have IRB
approval, we needed a way to remove the faces of the students so that
we could freely distribute the videos. 

Even with a subset of features, there is still much research to be done
in the fields of activity recognition and motion classification \cite{machine_perception}.
Some research involves using action banks \cite{action_bank} which involves
creating very large banks of actions to create feature vectors 
to classify with support vector machines (SVM). Action banks
can be though of as an extension to object banks. Other techniques involve 
using principal component analysis (PCA) \cite{face_recog_book} to 
detect what features actually contribute to training the classifier.
Regardless of the technique though, there is an obvious need for
research in these fields.

To support further investigation in these fields, we want to make the videos that are available
in the AOLME database more accessible to researchers interested in 
activity recognition and motion classification. In order to do this, 
we need to clean the dataset by removing the faces to satisfy the requirements
of the IRB.

Face recognition is a well studied field because it has applications in 
social media, law enforcement, education, and a myriad of others. There is a 
variety of texts that are available to use as references for this paper such
as \cite{face_recog_book} \cite{kernel_learning} \cite{machine_face_recog} 
as well as many software tools such as OpenCV \cite{opencv} and scikit-image
\cite{scikit-image}. There is, however work to be done in the areas of
big data. Most applications focus on real-time face tracking, or finding
faces in a few images and therefore there is room for research in the area
face detection using big data processing techniques. In this paper, we
would like to address this topic by looking at using OpenCV, scikit-image, 
Spark and Storm software packages for processing images in very large
video datasets.

The rest of the paper structured as follows: in the theory section we 
present some of the current techniques for facial recognition using 
machine learning to extract features that are necessary for face 
classification. We go over the theory of feature extraction and encoding, 
image-based classifiers, adaptive similarity threshold and temporal filtering. 
In the methodology section, we explain how we can take some of these
techniques and apply them using big data processing techniques. 
We explore experimental set-up and software utilization. It is 
important here that we show how we can leverage existing techniques
to explore some of this large video data effectively using distributed
files systems such as the Hadoop Distributed File System (HDFS) \cite{hadoop}. 
Finally, we explain in the conclusions section how well our 
implementation of face recognition worked with respect to 
performance, accuracy, and feasibility for use with AOLME to
 anonymize the videos for analysis by researchers who do not have
IRB access.  

%----------------------------------------------------------------------
% SECTION THEORY 
%----------------------------------------------------------------------
\section{Theory}
\PARstart As mentioned in the introduction, face detection is a well 
studied field because of all the applications that involved with 
being able to determine where faces are in data. Face recognition
is nothing more than a subset of a well researched area known as 
pattern recognition. In the subsequent section, we introduce some
of the basic techniques for extracting faces using statistical 
learning theory. Most of the information about this area comes
references \cite{face_recog_book} and \cite{kernel_learning}.
\subsection{Face Detection In Images}
There are many techniques for classifying images such as neural
networks (NN) \cite{face_neural_net}, template matching \cite{face_template}, support vector machines (SVM) \cite{face_svm}, 
nonnegative matrix factorization (NMF) \cite{face_mat_fact}, subspace analysis \cite{face_subspace_analysis}, 
elastic graph matching \cite{face_elastic_bunch}, and finally principal component analysis (PCA) \cite{face_kpca}. 
Though there are many different types of face recognition, in this paper we 
will focus directly on using Eigenfaces and SVMs. It should be noted 
that some algorithms are very good at determining to whom a face
belongs, but we are only interested in answering the question "
is this a face?". In fact, it is more important to uphold the IRB
to have a 100\% accuracy rate at the cost of some false positives.

\subsubsection{Eigenfaces}
Computing the "eigenfaces" is nothing more than using PCA \cite{eigen_faces}. 
The goal with computing the so-called eignefaces is to lower the dimensionality
of an image (or picture) to only its principal components, that is to say
that only relatively small number of dimensions should be used
to sufficiently identify pictures \cite{eigen_faces}. To put this in 
terms of mathematics, we wish to find the eigenvectors of the covariance
matrix for any given set of images \cite{eigen_face_recognition}.



\subsection{Face Removal}


%----------------------------------------------------------------------
% SECTION METHODOLOGY 
%----------------------------------------------------------------------
\section{Methodology and Experiments} 
\subsection{Machine Learning} 
\subsection{Distributed Computing}

%----------------------------------------------------------------------
% SECTION Conclusion 
%----------------------------------------------------------------------
\section{Conclusion}
\PARstart Conclusion to paper

\bibliography{project.bib}
\bibliographystyle{ieeetr}

\end{document}
